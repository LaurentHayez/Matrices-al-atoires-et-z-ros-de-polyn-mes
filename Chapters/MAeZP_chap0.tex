%-----------------------------------------------------------------------%
%______//------ Matrices aléatoires et zéros de polynômes ------\\______%
%______||------                Chapitre 0                 ------||______%
%______\\------  ------//______%
%-----------------------------------------------------------------------%

\setcounter{chapter}{-1}

\chapter{Résumé}

\begin{center}
  \begin{tikzpicture}[scale=0.8]
    \node[cloud, draw, cloud puffs=10, cloud puff arc=120, aspect=1, inner ysep=1em, text width=2cm, text
    centered, ForestGreen] (1) at (0,0){Dirac\\ Mécanique quantique};
    \node[draw, text width = 5cm, text centered] (2) at (12,0) {Problème de \textsc{Kadison-Singer}, 1959\\
      Analyse fonctionnelle};
    \node[draw, text width = 5cm, text centered] (3) at (12,-5) {Traduction en algèbre linéaire (1970-2002)};
    \node[draw, text width = 5cm, text centered] (4) at (6,-10) {Résolution en 2013 par 3 informaticiens
      (algèbre linéaire + calcul à plusieurs variables + probabilités élémentaires)};

    \node[draw, text width = 5cm, text centered] (5) at (0,-15) {Problèmes de théorie des graphes};
    \node[draw, text width = 5cm, text centered] (6) at (12,-15) {Problèmes d'informatique théorique \og
      problème du voyageur de commerce\fg{}};

    \draw[->,>=latex, line width=0.7pt] (1.east) -- (2.west);
    \draw[->,>=latex, line width=0.7pt] (2.south) -- (3.north);
    \draw[->,>=latex, line width=0.7pt] (3.west) -| (4.north);
    \draw[->,>=latex, line width=0.7pt] (4.east) -| (6.north);
    \draw[->,>=latex, line width=0.7pt] (4.west) -| (5.north) node[midway, above right]{Méthode};
  \end{tikzpicture}
\end{center}





%%% Local Variables:
%%% mode: latex
%%% TeX-master: "../MAeZP_cours.tex" 
%%% End: