%-----------------------------------------------------------------------%
%______//------ Matrices aléatoires et zéros de polynômes ------\\______%
%______||------                Chapitre 1                 ------||______%
%______\\------  ------//______%
%-----------------------------------------------------------------------%

\chapter{De la mécanique quantique à l'analyse fonctionnelle}

\section{Mécanique quantique}

 Paul Adrien Maurice \textsc{Dirac} (1902 - 1984) physicien anglais d'origine valaisanne, a écrit en 1930 les
 \og Principles of quantum mechanics\fg{} (plusieurs fois ré-édités).

 Principes:
 \begin{itemize}
 \item Les états d'un système physique sont représentés par les vecteurs-unités d'un espace de Hilbert $\H$.
   \begin{exs}
     \begin{itemize}
     \item Une particule libre à une dimension: $\H = L^2(\R)$.
       
     \item Une particule libre à deux dimensions: $\H = L^2(\R^2)$.
     \end{itemize}
   \end{exs}
   
 \item Les grandeurs physiques (\og observables\fg{}) sont des opérateurs auto-adjoints $(T = T^\ast)$ sur
   $\H$.
   \begin{exs}
     \begin{itemize}
     \item Sur $L^2(\R)$, l'\emph{\index{opérateur de position}opérateur de position} $P$ est la
       multiplication par $x$ sur $L^2(\R)$.
     \item L'\emph{\index{opérateur de moment}opérateur de moment} (ou impulsion) $Q = \frac{1}{i} \frac{d}{dx}$.
     \end{itemize}
   \end{exs}
   
 \item Ce qu'on peut observer (en laboratoire) est la \emph{probabilité} que la valeur d'une observable sur un
   état donné, soit comprise entre deux valeurs $a$ et $b$.
   \begin{exs}
     \begin{itemize}
     \item Soit $\psi \in L^2(\R)$, $\|\psi\| = 1$, c'est-à-dire 
       \[ \int_{-\infty}^\infty |\psi(x)|^2dx = 1.\]
       La probabilité que la position d'une particule dans l'état $\psi$, soit entre $a$ et $b$ est 
       \[ \int_a^b |\psi(x)|^2 dx. \]
       En effet, le fait que $\|\psi\| = 1$ nous dit que $\psi$ est une densité de probabilité.
     \end{itemize}
   \end{exs}
 \end{itemize}

 Ces trois principes sont parfois appelés les \og \index{axiomes de la mécanique quantique}axiomes de la
 mécanique quantique\fg{}.

 Dans le chapitre 18 de son livre, \og Probability amplitudes\fg{}, Dirac donne une recette pour obtenir ces
 probabilités.

 \begin{defi}
   Deux observables $S$ et $T$ sont dit \emph{\index{observables compatibles}compatibles} si $ST = TS$.
 \end{defi}

 \begin{exs}
   \begin{enumerate}
   \item Les observables $P$ et $Q$ précédemment définis ne sont pas compatibles. En effet, 
     \[ PQ - QP = \frac{-1}{i} Id, \]
     qui est  la \emph{\index{relation d'indétermination de Heisenberg}relation d'indétermination de
       Heisenberg}.
     
   \item Pour une particule libre à deux dimensions, posons $P_x$ l'opérateur de multiplication par la
     première variable $x$ sur $L^2(\R)$ ($P_x f(x,y) = xf(x,y)$) et $P_y$ l'opérateur de multiplication par la
     deuxième variable $y$. Ces deux opérateurs sont compatibles, 
     \[ P_xP_y = P_yP_x. \]
   \end{enumerate}
 \end{exs}

 La recette de Dirac est la suivante:
 \begin{enumerate}
 \item considérer un ensemble maximal d'observables deux à deux compatibles,
 \item spécifier les probabilités associées sur les observables dans un état quantique donné,
 \item étendre ces probabilités à toutes les observables, même non compatibles.
 \end{enumerate}


 \section{$C^\ast$-algèbres}
\label{sec:c-star-algebres}

Si $\H$ est un espace de Hilbert, on note $\bh$ l'\index{espace des opérateurs linéaires bornés}espace des
opérateurs linéaires bornés (donc continus) de $\H$ vers $\H$, avec la norme opérateur 
\[ \|T\| = \sup_{\|x\| \leq 1} \|Tx\|. \]
L'espace $\bh$ est une \emph{\index{algèbre de Banach}algèbre de Banach} 
\[ \|ST\| \leq \|S\|\|T\|. \]
Si $T \in \bh$, $T^\ast \in \bh$ est défini par 
\[ \left \langle T \xi\, ,\, \eta \right \rangle = \left \langle \xi\, ,\, T^\ast \eta \right \rangle  \quad
  \forall \xi, \eta \in \H.\]


\begin{defi}
  \begin{itemize}
  \item Une \emph{\index{sous-algèbre}sous-algèbre} est un sous espace vectoriel qui est stable pour la
    multiplication.
  \item Une \emph{\index{$\ast$-sous-algèbre}$\ast$-sous-algèbre} $A$ a la propriété $T \in A \Rightarrow
    T^\ast \in A$ et est fermée pour la norme opérateur.
  \end{itemize}

\end{defi}

\begin{defi}
  Une \emph{\index{$C^\ast$-algèbre}$C^\ast$-algèbre} est une $\ast$-sous-algèbre fermée de $\bh$.
\end{defi}

\begin{exs}
  \begin{enumerate}
  \item $\bh$, $\C \mathbbm{1} = \{\lambda \mathbbm{1},\ \lambda \in \C\}$, $\{0\}$, $\mathcal{K}(\H)$ l'ensemble des opérateurs
    compacts.
  \item Soit $X$ un espace topologique compact, et $C(X)$ l'ensemble des fonctions continues de $X$ dans $\C$
    muni de la norme 
    \[ \|f\|_{\infty} = \max_{x \in X} |f(x)| \]
    et de l'involution 
    \[ f^\ast(x) = \overline{f(x)}. \]
    On prend sur $X$ une mesure de probabilité $\mu$ telle que $\mu(U) > 0$ pour tout ouvert non vide de
    $X$. \textit{L'exemple a garder en tête est $[0,1]$ avec la mesure de Lebesgue.} Ainsi $\H = L^2(X, \mu)$
    est un espace de Hilbert. Si on multiplie une fonction de $L^2$ par une fonction continue, elle reste dans
    $L^2$, ainsi 
    \[ \pi \colon C(X) \to \bh,\ f \mapsto \text{multiplication par $f$ sur } \H. \]
    \begin{exercice}
      Avec $\|\pi(f)\| = \|f\|_\infty$, $C(X)$ est une $C^\ast$-algèbre.
    \end{exercice}
  \end{enumerate}
\end{exs}

\begin{theo}[Gelfand, 1940]
  Soit $A$ une $C^\ast$-algèbre commutative à unité $(1 \in A)$. Il existe un espace compact $X$, unique à
  homéomorphisme près, tel que $A \simeq C(X)$.
\end{theo}


\begin{defi}
  Un opérateur $T \in \bh$ est \emph{\index{opérateur positif}positif} $(T \geq 0)$ si les conditions équivalentes
  suivantes sont satisfaites:
  \begin{enumerate}
  \item pour tout $\xi \in \H$, $\left \langle T\xi\, ,\, \xi \right \rangle \geq 0$,
  \item il existe $S \in \bh$ tel que $T = S^\ast S$,
  \item $T = T^\ast$ et $\Sp(T) \subset [0, + \infty[$ (Rappel: le spectre de $T$ $\Sp(T) = \{\lambda
      \in \C\ \colon \ T - \lambda \mathbbm{1} \text{ n'est pas inversible}\}$, $\Sp(T)$ est un compact non
      vide de $\C$ et $\Sp(T) \subseteq B(0, \|T\|)$).
  \end{enumerate}
\end{defi}


\begin{defi}
  Si $A$ est une $C^\ast$-algèbre à unité, un \emph{\index{état}état} sur $A$ est une forme linéaire $\phi: A
  \to \C$ telle que
  \begin{enumerate}
  \item $\phi(1) = 1$,
  \item $\phi(T^\ast T) \geq 0$ pour tout $T \in A$.
  \end{enumerate}
  On note $S(A)$ l'ensemble des états sur $A$.
\end{defi}


\begin{ex}
  Soit $\xi \in \H$ tel que $\|\xi\| = 1$, alors pour $T \in \bh$
  \[ \phi(T) = \left \langle T\xi\, ,\, \xi \right \rangle \]
  est un \emph{\index{état vectoriel}état vectoriel} et 
  \[ \phi(1) = \|\xi\|^2 = 1,\ \phi(T^\ast T) = \left \langle T^\ast T \xi\, ,\, \xi \right \rangle  = \left
      \langle T \xi\, ,\, T \xi \right \rangle  = \|T \xi\|^2 \geq 0 \]
  et ainsi $S(A) \neq \emptyset$.
\end{ex}

\begin{prop}
  $S(A)$ est une partie convexe de la boule-unité du dual $A^\ast$ (ici $A^\ast$ est l'ensemble des formes
  linéaires continues sur $A$).
\end{prop}

\begin{preuve}
  \begin{enumerate}
  \item Soient $\phi_1, \phi_2 \in S(A)$, soit $t \in [0,1]$. On doit montrer que 
    \[ t \phi_1 + (1-t)\phi_2 \in S(A). \]
    On a 
    \[ (t \phi_1 + (1-t)\phi_2)(\mathbbm{1}) = 1,\quad (t \phi_1 + (1-t)\phi_2)(T^\ast T) \geq 0. \]
    
  \item Si $\phi \in S(A)$, on doit montrer que $\|\phi\| \leq 1$, c'est-à-dire 
    \[ |\phi(t)| \leq 1 \text{ si } \|T\| \leq 1. \]
    Si $\|T\| \leq 1$, alors $\un - T^\ast T \geq 0$ car 
    \[ \left \langle (\un - T^\ast T)\xi\, ,\, \xi \right \rangle  = \|\xi\|^2 - \|T \xi\|^2 \geq 0 \]
    puisque $\|T\| \leq 1$. On peut encore écrire 
    \[ \un - T^\ast T = S^\ast S \]
    pour $S \in A$. Ainsi 
    \[ 1 - \phi(T^\ast T) = \phi(\un - T^\ast T) = \phi(S^\ast S) \geq 0 \implies \phi(T^\ast T) \leq 1. \]
    L'application $A \times A \to \C$, $(x, y) \mapsto \phi(y^\ast x)$ vérifie l'inégalité de Cauchy-Schwartz 
    \[ |\phi(y^\ast x)|^2 \leq \phi(y^\ast y) \phi(x^\ast x)\, \forall x, y \in A. \]
    Pour $x = T$, $y = \un$, 
    \[ |\phi(T)|^2 \leq \phi(\un) \phi(T^\ast T) \leq 1. \]
  \end{enumerate}
\end{preuve}

\begin{rem}
  Comme $1 = \phi(\un) \leq \|\phi\|$, on a que $S(A)$ est contenu dans la sphère unité de $A^\ast$.
\end{rem}

\begin{defi}
  Soit $K$ un convexe dans un espace vectoriel réel ou complexe. Un point $K$ est \emph{\index{point
      extrême}extrême} dans $K$ si $x$ n'est pas intérieur à un segment contenu dans $K$, c'est-à-dire si $x =
  tx_1 + (1-t)x_2$ avec $0 < t < 1$, $x_1, x_2 \in K$, alors $x = x_1 = x_2$.

  \begin{center}
    \begin{tikzpicture}
      \node[red] (A) at (0,0) {$\bullet$};
      \node[red] (B) at (3, -2) {$\bullet$};
      \node[red] (C) at (-2, -3) {$\bullet$};
      \draw (A.center) -- (B.center) -- (C.center) -- (A.center);
      \draw (A) node[red]{$\bullet$};
      \draw (B) node[red]{$\bullet$};
      \draw (C) node[red]{$\bullet$};
      \draw[ForestGreen, line width=0.7pt] (0, -2) -- (1, -1.5) node[midway]{$\bullet$};
      \draw[ForestGreen, line width=0.7pt] (-1, -1.5) -- (-0.5, -0.75) node[midway]{$\bullet$};
      \draw (8, -1.5) node{\textcolor{red}{$\bullet$} = points extrêmes};
    \end{tikzpicture}
  \end{center}

  Un point extrême de $S(A)$ est
  appelé un \emph{\index{état pur}état pur}.
\end{defi}

\begin{exs}
  \begin{enumerate}
  \item On montre que tout état vectoriel de $\bh$ est pur. On verra qu'il y a des états purs non vectoriels.
  \item Soit $A = C(X)$, alors $S(A)$ s'identifie avec $\mathrm{Prob}(X)$ l'espace des mesures de probabilités
    sur $X$ (un des théorèmes de représentation de \textsc{Riesz}). \qedhere
  \end{enumerate}
\end{exs}

Les états purs s'identifient aux points extrêmes de $\mathrm{Prob}(X)$, c'est-à-dire aux mesures de Dirac
$\delta_x$, définie par (pour $A \subset X$)
\[ \delta_x(A) =
  \begin{cases}
    1 & \text{si } x \in A,\\
    0 & \text{sinon.}
  \end{cases}
\]


{\LARGE Manque le cours du 28 février}

Problème: un état pur sur une MASA a-t-il une extension unique à $\bh$?

Le résultat principal de Kadison \& Singer (1959) est le théorème suivant, qu'on admet.

\begin{theo}
  Pour la MASA diffuse $L^{\infty}[0,1]$, la réponse est non.
\end{theo}

L'article laisse ouvert le cas de la MASA discrète $\ell^\infty(\N)$.

\paragraph{\index{Problème de Kadison\&Singer}Problème de Kadison\&Singer:} la MASA $\ell^\infty(\N)$ a-t-elle l'extension unique des états purs?

Kadison \& Singer inclinent à penser que la réponse doit être non (en fait c'est oui).

\begin{prop}
  Pour $k \in \N$, notons $\phi_k$ l'état pur de $\ell^\infty(\N)$ donné par $\phi_k(\underbrace{(a_n)_{n >
      0}}_{\in \ell^\infty(\N)}) = a_k$. Alors l'extension unique de $\phi_k$ à $\B(\ell^2(\N))$ est 
  \[ T \mapsto \left \langle T e_k\, ,\, e_k \right \rangle = T_{kk}. \]
\end{prop}

\begin{preuve}
  \begin{enumerate}
  \item Si $\psi$ est un état pur non vectoriel de $\bh$, alors $\psi$ est nul sur l'espace des opérateurs
    compacts (en particulier sur l'espace des opérateurs de rang fini). Ici, 
    \[ p_k = (0, \ldots, 0, \underbrace{1}_{k\text{-ème}}, 0, 0, \ldots) \in \ell^\infty(\N) \]
    et ainsi on a $\phi_k(p_k) = 1$. Donc une extension de $\phi_k$ à $\B(\ell^2(\N))$ est un état vectoriel,
    disons 
    \[ T \mapsto \left \langle T \xi\, ,\, \xi \right \rangle ,\ \|\xi\| = 1. \]
    
  \item À voir: $\xi = \lambda e_k$ avec $|\lambda| = 1$ (où $e_k$ est $p_k$ mais vu dans $\ell^2(\N)$ (on le
    voit une fois comme matrice diagonale, et une fois comme suite)). Si $T = (a_n)_{n > 0}$, 
    \[ \left \langle T \xi\, ,\, \xi \right \rangle = \sum_{n = 1}^{\infty} a_n |\xi_n|^2. \]
    Mais $T \xi = \phi_k(T) = a_k$. Ainsi, 
    \[ \sum_{n\neq k}^{}a_n |\xi_n|^2 + a_k(|\xi_k|^2 - 1) = 0\ \forall (a_n)_{n > 0} \in \ell^\infty(\N). \]
    Donc $\xi_n = 0$ si $n \neq k$; $|\xi_k| = 1$ et donc $\xi = \lambda e_k$, $|\lambda| = 1$.
  \end{enumerate}
\end{preuve}

\begin{defi}
  L'\index{application diagonale }application diagonale 
  \[ \mathrm{diag} \colon \B(\ell^2(\N)) \to \ell^\infty(\N),\ T \mapsto (T_{nn})_{n > 0} \]
  Intuition: on a un tableau carré, et on garde seulement la diagonale.
\end{defi}

\begin{exs}
  \begin{itemize}
    
  \item $\mathrm{diag}(\mathbbm{1}) = (1, 1, \ldots)$
  \item Si $T \geq 0$, $\mathrm{diag}(T) = \left( \underbrace{\left \langle Te_n\, ,\, e_n \right
        \rangle}_{\geq 0}  \right)_{n > 0}$.
  \end{itemize}
\end{exs}

Si $\phi$ est un état sur $\ell^\infty(\N)$, alors $\phi \circ \mathrm{diag}$ est un état sur $\B(\ell^2(\N))$
qui étend $\phi$. Ceci veut dire que pour étendre $\phi$ à $\B(\ell^2(\N))$, on n'a pas besoin du théorème de
Hahn-Banach. Grâce à cette observation, le problème de Kadison-Singer se reformule de la façon suivante.

\paragraph{\index{Problème de Kadison\&Singer}Problème de KS:} Si $\phi$ est un état pur de $\ell^\infty(\N)$, alors $\phi \circ \mathrm{diag}$
est l'unique extension de $\phi$ à $\B(\ell^2(\N))$.




%%% Local Variables:
%%% mode: latex
%%% TeX-master: "../MAeZP_cours.tex" 
%%% End: