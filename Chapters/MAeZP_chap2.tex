%-----------------------------------------------------------------------%
%______//------ Matrices aléatoires et zéros de polynômes ------\\______%
%______||------                Chapitre 2                 ------||______%
%______\\------  ------//______%
%-----------------------------------------------------------------------%

\chapter{De l'analyse fonctionnelle à l'algèbre linéaire}
\label{sec:chap2}


\section{Conjecture de pavage}
\label{sec:conjecture-de-pavage}

En 1979, le mathématicien américain Joel \textsc{Anderson} propose la \emph{\index{conjecture de
    pavage}conjecture de pavage}

\begin{conj}
  Pour tout $\epsilon > 0$, il existe $r \in \N$ tel que pour tout $T \in \B(\ell^2(N))$ (ou $M_m(\C)$) avec
  $\mathrm{diag}(T) = 0$, il existe $Q_1, \ldots, Q_r$ des projecteurs diagonaux avec 
  \[ \sum_{i=1}^{r} Q_i = \mathbbm{1} \]
  et 
  \[ \|Q_iTQ_i\|  \leq \epsilon \|T\|, \ \forall i \in \{1, \ldots, r\}. \]
\end{conj}

Pour rappel, un \index{projecteur}projecteur $P$ est tel que 
\[ P = P^2 = P^\ast. \]
C'est donc une projection orthogonale sur un certain sous-espace fermé de $\H$. Un \index{projecteur
  diagonal}projecteur diagonal a pour coefficients $a_{ii} \in {0, 1}$ et $a_{ij} = 0$ pour $i \neq j$.

Il y a une bijection entre les projecteurs diagonaux et les parties de $\N$. Ainsi 
\[ \sum_{i = 1}^{r} Q_i = \mathbbm{1} \]
signifie qu'on partitionne $\N$ en $r$ parties. Les $Q_1, \ldots, Q_r$ donnent une décomposition de
$\ell^2(\N)$ en $r$ blocs

\begin{center}
  \begin{tikzpicture}[xscale=1.3]
    % \draw[gray!70, opacity=0.5, dashed] (-1,1) grid (6, -7);  % help grid
    \draw (0,0) node {$Q_1 T Q_1$};
    \draw (2, -2) node {$Q_2 T Q_2$};
    \draw (6, -6) node {$Q_r T Q_r$};
    \draw (2, 0) node{$\ast$};
    \draw (0, -2) node{$\ast$};
    \draw (-2, 0) node{\textcolor{ForestGreen}{$Q_1$}};
    \draw (-2, -2) node{\textcolor{ForestGreen}{$Q_2$}};
    \draw (-2, -6) node{\textcolor{ForestGreen}{$Q_r$}};
    \draw (0, 1)  -| (-1, -7) -- (0, -7);
    \draw (6, 1)  -| (7, -7) -- (6, -7);
    \draw[ForestGreen] (1, 1) -- (1, -3) -- (5, -3) -- (5, -7);
    \draw[ForestGreen] (-1, -1) -- (3, -1) -- (3, -5) -- (7, -5);
  \end{tikzpicture}
\end{center}

La conjecture dit que la condition $\mathrm{diag}T = 0$ implique que les blocs sont de norme petite.

\begin{ex}
Soit $S: \ell^2(\N) \to \ell^2(\N)$ l'\index{opérateur de décalage unilatéral}opérateur de décalage
  unilatéral (\og unilateral shift\fg{}) défini par $Se_n = e_{n+1}$ ou
  $S(a_1, a_2, \ldots) = (0, a_1, a_2, \ldots)$. Alors
  \[ S =
    \begin{pmatrix}
      0 &   &        &   & \\
      1 & 0 &        &   & \\
        & 1 & 0 &   & \\
        &   & \ddots & \ddots  & \\
        &   &        & 1 & 0
    \end{pmatrix}
\] On prend la partition de $\N = (2\N) \cup (2\N + 1)$. Soit $Q_1$ la
  projection sur $\ell^2(2\N)$ et $Q_2$ la projection sur $\ell^2 (2\N + 1)$. Alors 
  \[ S =
    \begin{pmatrix}
      0 = Q_1Sq_1 & \ast \\ \ast & 0 = Q_2SQ_2
    \end{pmatrix}
  \]
  L'opérateur $S$ vérifie la conjecture de pavage (\og tiling conjecture\fg{}) avec $\epsilon = 0$ et $r = 2$.
\end{ex}

\begin{prop}
  Si la conjecture de pavage est vraie, alors 
  $ \ell^\infty(\N) $ a l'extension unique des états purs.
\end{prop}

\begin{preuve}
  Soit $\phi$ un état pur de $\ell^\infty(\N)$, $\psi$ une extension de $\phi$ à $\B(\ell^2(\N))$. On veut
  montrer que 
  \[ \psi = \phi \circ \mathrm{diag}. \]
  C'est-à-dire, pour tout $T \in \B(\ell^2(N))$, 
  \[ \psi(T) = \phi(\mathrm{diag}(T)). \]
  En remplaçant $T$ par $T - \mathrm{diag}(T)$, on obtient un opérateur de diagonale nulle. On doit donc
  montrer que si $\mathrm{diag} T = 0$, alors $\psi(T) = 0$, qui est ce qu'on va démontrer.

  On montre que pour tout $\epsilon > 0$, $|\psi(T)| \leq \epsilon \|T\|$. Par la conjecture de pavage, on
  trouve $r \in \N$ et des projecteurs diagonaux $Q_1, \ldots, Q_r$ tels que 
  \[ \sum_{i = 1}^{r} Q_i = \mathbbm{1} \]
  et 
  \[ \|Q_iTQ_i\| < \epsilon \|T\|. \]
  On utilise à présent le fait que $\phi$ est un état pur: $\phi$ est multiplicatif sur $\ell^\infty(\N)$
  (c'est-à-dire que $\phi(ST) = \phi(S)\phi(T)$ si $S, T \in \ell^\infty(\N)$). La raison est que
  $\ell^\infty(\N) = C(X)$ l'espace des fonctions continues sur $X$ car $\ell^\infty$ est une
  $C^\ast$-algèbre commutative à unité (c'est le théorème de \textsc{Gelfand}). Les états purs de $C(X$ sont
  les évaluations aux points, elles sont multiplicatives (ici $X = B\N$ est le compactifié de
  \textsc{Stone-\v{C}ech} de $\N$, c'est la plus grosse compactification de $\N$).
  Alors 
  \[ \phi(Q_i) = \phi(Q_i^2) = \phi(Q_i)^2 \implies \phi(Q_i) \in \{0,1\}. \]
  De plus, 
  \[ 1 = \phi(\mathbbm{1}) = \phi \left(\sum_{i = 1}^{r} Q_i\right) = \sum_{i = 1}^{r}\phi(Q_i). \]
  Puisque $\phi(Q_i) \in \{0,1\}$ et qu'on écrit $1$ comme somme d'éléments de $\{0,1\}$, il existe un unique
  indice $i_0$ avec $\phi(Q_{i_0}) = 1$ et $\phi(Q_i) = 0$ si $i \neq i_0$.

  Alors 
  \[ \psi(T) = \psi \left( \left(\sum_{i=1}^{r} Q_i\right) T \left(\sum_{j = 1}^{r} Q_j\right)\right) =
    \sum_{i,j=1}^{r} \psi(Q_iTQ_j). \]
  Si on sait que dans ces $r^2$ termes, le seul terme non nul est $\psi(Q_{i_0}TQ_{i_0})$, alors 
  \[ |\psi(T)| = |\psi(Q_{i_0}TQ_{i_0})| \leq \|Q_{i_0}TQ_{i_0}\| \leq \epsilon \|T\| \]
  comme on voulait, où la dernière inégalité suit de la conjecture de pavage, et la première du fait que
  $\psi$ est un état pur, donc de norme $1$ (note: à vérifier).

  Par Cauchy-Schwartz, 
  \begin{align*}
    |\psi(Q_iTQ_j)| &= |\psi\left((T^\ast Q_i)^\ast Q_j\right)\\
    &\leq \psi \left((T^\ast Q_i)^\ast (T^\ast Q_i)\right)^{1/2} \underbrace{\psi(\underbrace{Q_j^\ast Q_j}_{=Q_j}}_{\phi(Q_j)}) ^{1/2}\\
    &= 0 \text{ si } j \neq i_0.
  \end{align*}
  De même, 
  \[ |\psi(Q_i T Q_j)| = 0 \]
  si $i \neq i_0$ par un argument similaire. Le seul terme non nul restant est donc $\psi(Q_{i_0}TQ_{i_0})$,
  comme on le souhaitait.
\end{preuve}



\section{Conjecture de \textsc{Weaver} (2004)}
\label{sec:conj-de-weaver}
(Approche de Terry \textsc{Tao}, 2013)

\begin{conj}[\index{conjecture de \textsc{Weaver}}conjecture de \textsc{Weaver}]
  \label{conj:conjecture-weaver}
  On fixe des entiers $d, m, r \geq 2$ et une constante $c > 0$. Soient $A_1, \ldots, A_d \in M_m(\C)$, $A_i
  \geq 0$, $\mathrm{rang}(A_i) = 1$ avec $\|A_i\| \leq C$ pour tout $i = 1, \ldots d$ et 
  \[ \sum_{i = 1}^{d}A_i = \mathbbm{1}_m. \]
  Alors il existe une partition $\{S_1, S_2, \ldots, S_r\}$ de $\{1, \ldots, d\}$ telle que 
  \[ \left\| \sum_{i \in S_j}^{} A_i \right\| \leq \left(\sqrt{\frac{1}{r}} + \sqrt{C}\right)^2 \]
  pour $j = 1, \ldots, r$.
\end{conj}

Pour cette conjecture, il faut penser à $d$ et $m$ grands, et puisque les matrices sont de rang $1$ et qu'on
veut avoir leur somme égale à $\mathbbm{1}$, on a besoin d'au moins $m$ telle matrices, i.e., $d \geq m$. De
plus il faut penser à $r$ petit (cas extrême, $r = 2$).

Si $A \geq 0$ et $\mathrm{rang}(A) = 1$ alors $A$ est un multiple positif d'un projecteur orthogonal de rang
$1$. C'est-à-dire qu'il existe $\xi \in \C^m$ tel que $\|\xi\| = 1$ et $\lambda > 0$ tel que 
\[ A(v) = \lambda \left \langle v\, ,\, \xi \right \rangle \xi. \]

Si les $A_i$ ont des images $2$ à $2$ orthogonales (sans supposer que leur somme vaut $\mathbbm{1}$ comme dans
la conjecture), alors la somme se décompose par blocs, 
\[ \sum_{}^{} A_i =
  \begin{pmatrix}
    A_1 & 0 & \cdots & 0\\ 0 & A_2 &  & 0\\ \vdots & \vdots & \ddots & \vdots \\ 0 & 0 & \cdots & A_k
  \end{pmatrix} \implies \left \| \sum_{}^{} A_i \right \| = \max \|A_i\| \leq C.
  \]

Ainsi, ce que veut dire la conjecture, c'est que ces matrices sont \textit{presque} orthogonales. En d'autres
termes, on peut partitionner l'ensemble des indices de façon à ce que pour chaque classe de la partition, les
$A_i$ soient d'images \textit{quasiment} orthogonales.

\begin{prop}
  La conjecture de \textsc{Weaver} implique la conjecture de pavage.
\end{prop}



{\LARGE Manque le cours du 14 mars}


\section{Matrices aléatoires}
\label{sec:matrices-aleatoires}

Référence classique: M.L. \textsc{Mehta}, Random matrices, Academic Press 1991.

\begin{defi}
  Une \emph{\index{matrice aléatoire}matrice aléatoire} est un tableau carré 
  \[ \begin{pmatrix}
      X_{11} & X_{12} & \cdots & X_{1n} \\
      \vdots & \vdots & & \vdots  \\
      X_{n1} & X_{n2} & \cdots & X_{nn}
    \end{pmatrix} \]
  où les $X_{ij}$ sont des variables aléatoires indépendantes à valeurs complexes.

  Le \index{polynôme caractéristique}polynôme caractéristique 
  \[ p_X(z) := \det(z \un_n - X) \]
  est un polynôme aléatoire et les valeurs propres sont des variables aléatoires au sens ordinaire.
\end{defi}

Typiquement, la théorie s'intéresse au comportement des valeurs propres pour $n \to \infty$.


\begin{ex}
Un modèle très simple: on choisit au hasard un nombre dans $\{1, \ldots, n\}$ (avec probabilité $1/n$), si le
résultat est $i$, on choisit le projecteur sur le $i$-ème vecteur de la base canonique de $\C^n$.
\end{ex}

\paragraph{Historique:} Eugene \textsc{Wigner} (1902 - 1995), physicien hongrois naturalisé américain, prix
Nobel en physique en 1963 pour ses travaux sur théoriques sur la structure des noyaux atomiques. À la fin des
années 1940, il propose l'hypothèse: pour modéliser un gros noyau d'uranium (par exemple $U^{239}$) au niveau
quantique, il faut s'intéresser au spectre d'une grosse matrice aléatoire $X$, auto-adjointe
(i.e. $\overline{X_{ij}} = X_{ji}$) où les $X_{ij}$ sont des variables gaussiennes iid.


\paragraph{Théorèmes de MSS:} voir photocopie.

Si $A$ est une matrice, on définit l'\index{espérance}espérance de $A$ par 
\[ \E[A] = \sum_{i}^{} p_i X_i \]
où les $X_i$ sont les valeurs possibles de $A$. Pour une loi continue, on aurait $(X, \mathcal{B}, \mu)
\xrightarrow{A} M_m(\C)$, on a 
\[ \E[A] = \int_X A(\omega) d\mu(\omega) \in M_m(\C). \]

\begin{preuve}[preuve fausse du théorème 1, mais inspirante]
  On a 
  \[ (\E[p_A])(z) = \E(\det(z \un_m - A)) = \det(z \un_m - \E[A]) = \det((z - 1)\un_m) = (z - 1)^m. \]
\end{preuve}

On a $\|A\|$ est la plus grande valeur propre de $A$ (car $A \geq 0$) est aussi la plus grande racine de $p_A$. Le théorème 2 se reformule de la
façon suivante.

\begin{theo}
  Pour au moins une réalisation des $A_i$, la plus grande racine de $p_A$ est inférieure à la plus grande
  racine de $\E[p_A]$.
\end{theo}


C'est une version non linéaire d'un principe de probabilité: si $X$ est une variable aléatoire à valeurs
réelles, alors pour au moins une réalisation, $X \leq \E[X]$.


\begin{prop}
  Les théorèmes 1 et 2 de MSS impliquent la conjecture de \textsc{Weaver} \ref{conj:conjecture-weaver}.
\end{prop}

\begin{preuve}
  Soient $A_1, \ldots, A_d \in M_m(\C)$ telles que 
  \[ \sum_{i = 1}^{d}A_i = \un_m, \]
  et $A_i \geq 0$ de rang 1, avec $\|A_i\| \leq C$.

  Pour $i  =1, \ldots, d$, on définit une variable aléatoire $\tilde{A_i}$ à valeurs dans $M_{mr}(\C)$, on
  écrit 
  \[ \C^{mr} = \C \oplus \C^m \oplus \cdots \oplus \C^m. \]
  On choisit $j \in \{1, 2, \ldots, r\}$ avec probabilité $\frac{1}{r}$ et on place $r A_i$ dans le $j$-ème
  facteur, i.e. le bloc $jj$ de la matrice $\tilde{A_i}$ vaut $rA_i$, et les autres blocs valent
  0. $\tilde{A_i}$ est une variable aléatoire à valeurs dans les opérateurs positifs de rang 1 dans
  $M_{mr}(\C)$. On pose 
  \[ \tilde{A} = \sum_{i = 1}^{d} \tilde{A_i}. \]
  On veut appliquer les théorèmes 1 et 2 de MSS à $\tilde{A}$. Vérifions l'hypothèse du théorème 1. On a 
  \begin{align*}
    \E[\tilde{A}]
    &= \sum_{i = 1}^{d} \E[\tilde{A_i}] \\
    &= \sum_{i = 1}^{d} \frac{1}{r} \sum_{j = 1}^{r}
    \begin{pmatrix}
      0 &        &    & && \\
        & \ddots & & && \\
        &        & 0 &      & && \\
        &        &   & rA_i & & &\\
        &        &   &      & 0 & & \\
        &        &   &      &   & \ddots & \\
        &        &   &      &   &        & 0     
      \end{pmatrix}\\
    &= \sum_{j = 1}^{r} \sum_{i = 1}^{d}
      \begin{pmatrix}
      0 &        &    & && \\
        & \ddots & & && \\
        &        & 0 &      & && \\
        &        &   & A_i & & &\\
        &        &   &      & 0 & & \\
        &        &   &      &   & \ddots & \\
        &        &   &      &   &        & 0  
      \end{pmatrix}\\
    &= \sum_{j = 1}^{r}
      \begin{pmatrix}
      0 &        &    & && \\
        & \ddots & & && \\
        &        & 0 &      & && \\
        &        &   & \un_m & & &\\
        &        &   &      & 0 & & \\
        &        &   &      &   & \ddots & \\
        &        &   &      &   &        & 0
      \end{pmatrix}\\
    &= \begin{pmatrix}
      \un_m &        &    & && \\
        & \ddots & & && \\
        &        & \un_m &      & && \\
        &        &   & \un_m & & &\\
        &        &   &      & \un_m & & \\
        &        &   &      &   & \ddots & \\
        &        &   &      &   &        & \un_m
      \end{pmatrix}\\
    &= \un_{mr}.
  \end{align*}
  De plus, $\|\tilde{A_i}\| = r \|A_i\| \leq rC$. Par les théorèmes 1 et 2, pour au moins une réalisation des
  $\tilde{A_i}$, on a 
  \[ \|\tilde{A}\| \leq \text{plus grande racine de } \E[p_{\tilde{A}}] \leq (1 + \sqrt{rC})^2. \]
  La partition $\{S_1, \ldots, S_r\}$ de $\{1, 2, \ldots, d\}$ est associée à cette réalisation, 
  \[ S_j = \left\{ i \in \{1, 2, \ldots, d\} \cdot j \text{ a été choisi au $i$-ème essai}\right\}. \]
  Alors 
  \[ r \left\| \sum_{i \in S_j}^{}A_i \right\| = \left\| \sum_{i \in S_j}^{} \tilde{A_j} \right\| \leq
    \left\| \sum_{i = 1}^{d} \tilde{A_i}\right\| = \|\tilde{A}\| \leq (1 + \sqrt{rc})^2. \]
  On divise par $r$, et on obtient 
  \[ \left\| \sum_{i \in S_j}^{}A_i \right\| \leq \left( \sqrt{\frac{1}{r}} + \sqrt{C}\right)^2. \]
\end{preuve}












%%% Local Variables:
%%% mode: latex
%%% TeX-master: "../MAeZP_cours.tex" 
%%% End: