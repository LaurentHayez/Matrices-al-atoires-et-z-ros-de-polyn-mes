%-----------------------------------------------------------------------%
%______//------ Matrices aléatoires et zéros de polynômes ------\\______%
%______||------                Chapitre 2                 ------||______%
%______\\------  ------//______%
%-----------------------------------------------------------------------%

\chapter{De l'analyse fonctionnelle à l'algèbre linéaire}
\label{sec:chap2}


\section{Conjecture de pavage}
\label{sec:conjecture-de-pavage}

En 1979, le mathématicien américain Joel \textsc{Anderson} propose la \emph{\index{conjecture de
    pavage}conjecture de pavage}

\begin{conj}
  Pour tout $\epsilon > 0$, il existe $r \in \N$ tel que pour tout $T \in \B(\ell^2(N))$ (ou $M_m(\C)$) avec
  $\mathrm{diag}(T) = 0$, il existe $Q_1, \ldots, Q_r$ des projecteurs diagonaux avec 
  \[ \sum_{i=1}^{r} Q_i = \mathbbm{1} \]
  et 
  \[ \|Q_iTQ_i\|  \leq \epsilon \|T\|, \ \forall i \in \{1, \ldots, r\}. \]
\end{conj}

Pour rappel, un \index{projecteur}projecteur $P$ est tel que 
\[ P = P^2 = P^\ast. \]
C'est donc une projection orthogonale sur un certain sous-espace fermé de $\H$. Un \index{projecteur
  diagonal}projecteur diagonal a pour coefficients $a_{ii} \in {0, 1}$ et $a_{ij} = 0$ pour $i \neq j$.

Il y a une bijection entre les projecteurs diagonaux et les parties de $\N$. Ainsi 
\[ \sum_{i = 1}^{r} Q_i = \mathbbm{1} \]
signifie qu'on partitionne $\N$ en $r$ parties. Les $Q_1, \ldots, Q_r$ donnent une décomposition de
$\ell^2(\N)$ en $r$ blocs

//Insérer figure matrice ici

La conjecture dit que la condition $\mathrm{diag}T = 0$ implique que les blocs sont de norme petite.

\begin{ex}
Soit $S: \ell^2(\N) \to \ell^2(\N)$ l'\index{opérateur de décalage unilatéral}opérateur de décalage
  unilatéral (\og unilateral shift\fg{}) défini par $Se_n = e_{n+1}$ ou
  $S(a_1, a_2, \ldots) = (0, a_1, a_2, \ldots)$. Alors
  \[ S = \text{insert matrix here} \] On prend la partition de $\N = (2\N) \cup (2\N + 1)$. Soit $Q_1$ la
  projection sur $\ell^2(2\N)$ et $Q_2$ la projection sur $\ell^2 (2\N + 1)$. Alors 
  \[ S =
    \begin{pmatrix}
      0 = Q_1Sq_1 & \ast \\ \ast & 0 = Q_2SQ_2
    \end{pmatrix}
  \]
  L'opérateur $S$ vérifie la conjecture de pavage (\og tiling conjecture\fg{}) avec $\epsilon = 0$ et $r = 2$.
\end{ex}

\begin{prop}
  Si la conjecture de pavage est vraie, alors 
  $ \ell^\infty(\N) $ a l'extension unique des états purs.
\end{prop}

\begin{preuve}
  Soit $\phi$ un état pur de $\ell^\infty(\N)$, $\psi$ une extension de $\phi$ à $\B(\ell^2(\N))$. On veut
  montrer que 
  \[ \psi = \phi \circ \mathrm{diag}. \]
  C'est-à-dire, pour tout $T \in \B(\ell^2(N))$, 
  \[ \psi(T) = \phi(\mathrm{diag}(T)). \]
  En remplaçant $T$ par $T - \mathrm{diag}(T)$, on obtient un opérateur de diagonale nulle. On doit donc
  montrer que si $\mathrm{diag} T = 0$, alors $\psi(T) = 0$, qui est ce qu'on va démontrer.

  On montre que pour tout $\epsilon > 0$, $|\psi(T)| \leq \epsilon \|T\|$. Par la conjecture de pavage, on
  trouve $r \in \N$ et des projecteurs diagonaux $Q_1, \ldots, Q_r$ tels que 
  \[ \sum_{i = 1}^{r} Q_i = \mathbbm{1} \]
  et 
  \[ \|Q_iTQ_i\| < \epsilon \|T\|. \]
  On utilise à présent le fait que $\phi$ est un état pur: $\phi$ est multiplicatif sur $\ell^\infty(\N)$
  (c'est-à-dire que $\phi(ST) = \phi(S)\phi(T)$ si $S, T \in \ell^\infty(\N)$). La raison est que
  $\ell^\infty(\N) = C(X)$ l'espace des fonctions continues sur $X$ car $\ell^\infty$ est une
  $C^\ast$-algèbre commutative à unité (c'est le théorème de \textsc{Gelfand}). Les états purs de $C(X$ sont
  les évaluations aux points, elles sont multiplicatives (ici $X = B\N$ est le compactifié de
  \textsc{Stone-\v{C}ech} de $\N$, c'est la plus grosse compactification de $\N$).
  Alors 
  \[ \phi(Q_i) = \phi(Q_i^2) = \phi(Q_i)^2 \implies \phi(Q_i) \in \{0,1\}. \]
  De plus, 
  \[ 1 = \phi(\mathbbm{1}) = \phi \left(\sum_{i = 1}^{r} Q_i\right) = \sum_{i = 1}^{r}\phi(Q_i). \]
  Puisque $\phi(Q_i) \in \{0,1\}$ et qu'on écrit $1$ comme somme d'éléments de $\{0,1\}$, il existe un unique
  indice $i_0$ avec $\phi(Q_{i_0}) = 1$ et $\phi(Q_i) = 0$ si $i \neq i_0$.

  Alors 
  \[ \psi(T) = \psi \left( \left(\sum_{i=1}^{r} Q_i\right) T \left(\sum_{j = 1}^{r} Q_j\right)\right) =
    \sum_{i,j=1}^{r} \psi(Q_iTQ_j). \]
  Si on sait que dans ces $r^2$ termes, le seul terme non nul est $\psi(Q_{i_0}TQ_{i_0})$, alors 
  \[ |\psi(T)| = |\psi(Q_{i_0}TQ_{i_0})| \leq \|Q_{i_0}TQ_{i_0}\| \leq \epsilon \|T\| \]
  comme on voulait, où la dernière inégalité suit de la conjecture de pavage, et la première du fait que
  $\psi$ est un état pur, donc de norme $1$ (note: à vérifier).

  Par Cauchy-Schwartz, 
  \begin{align*}
    |\psi(Q_iTQ_j)| &= |\psi\left((T^\ast Q_i)^\ast Q_j\right)\\
    &\leq \psi \left((T^\ast Q_i)^\ast (T^\ast Q_i)\right)^{1/2} \underbrace{\psi(\underbrace{Q_j^\ast Q_j}_{=Q_j}}_{\phi(Q_j)}) ^{1/2}\\
    &= 0 \text{ si } j \neq i_0.
  \end{align*}
  De même, 
  \[ |\psi(Q_i T Q_j)| = 0 \]
  si $i \neq i_0$ par un argument similaire. Le seul terme non nul restant est donc $\psi(Q_{i_0}TQ_{i_0})$,
  comme on le souhaitait.
\end{preuve}



\section{Conjecture de \textsc{Weaver} (2004)}
\label{sec:conj-de-weaver}
(Approche de Terry \textsc{Tao}, 2013)

\begin{conj}[\index{conjecture de \textsc{Weaver}}conjecture de \textsc{Weaver}]
  On fixe des entiers $d, m, r \geq 2$ et une constante $c > 0$. Soient $A_1, \ldots, A_d \in M_m(\C)$, $A_i
  \geq 0$, $\mathrm{rang}(A_i) = 1$ avec $\|A_i\| \leq C$ pour tout $i = 1, \ldots d$ et 
  \[ \sum_{i = 1}^{d}A_i = \mathbbm{1}_m. \]
  Alors il existe une partition $\{S_1, S_2, \ldots, S_r\}$ de $\{1, \ldots, d\}$ telle que 
  \[ \left\| \sum_{i \in S_j}^{} A_i \right\| \leq \left(\sqrt{\frac{1}{r}} + \sqrt{C}\right)^2 \]
  pour $j = 1, \ldots, r$.
\end{conj}

Pour cette conjecture, il faut penser à $d$ et $m$ grands, et puisque les matrices sont de rang $1$ et qu'on
veut avoir leur somme égale à $\mathbbm{1}$, on a besoin d'au moins $m$ telle matrices, i.e., $d \geq m$. De
plus il faut penser à $r$ petit (cas extrême, $r = 2$).

Si $A \geq 0$ et $\mathrm{rang}(A) = 1$ alors $A$ est un multiple positif d'un projecteur orthogonal de rang
$1$. C'est-à-dire qu'il existe $\xi \in \C^m$ tel que $\|\xi\| = 1$ et $\lambda > 0$ tel que 
\[ A(v) = \lambda \left \langle v\, ,\, \xi \right \rangle \xi. \]

Si les $A_i$ ont des images $2$ à $2$ orthogonales (sans supposer que leur somme vaut $\mathbbm{1}$ comme dans
la conjecture), alors la somme se décompose par blocs, 
\[ \sum_{}^{} A_i =
  \begin{pmatrix}
    A_1 & 0 & \cdots & 0\\ 0 & A_2 &  & 0\\ \vdots & \vdots & \ddots & \vdots \\ 0 & 0 & \cdots & A_k
  \end{pmatrix} \implies \left \| \sum_{}^{} A_i \right \| = \max \|A_i\| \leq C.
  \]

Ainsi, ce que veut dire la conjecture, c'est que ces matrices sont \textit{presque} orthogonales. En d'autres
termes, on peut partitionner l'ensemble des indices de façon à ce que pour chaque classe de la partition, les
$A_i$ soient d'images \textit{quasiment} orthogonales.

\begin{prop}
  La conjecture de \textsc{Weaver} implique la conjecture de pavage.
\end{prop}







%%% Local Variables:
%%% mode: latex
%%% TeX-master: "../MAeZP_cours.tex" 
%%% End: