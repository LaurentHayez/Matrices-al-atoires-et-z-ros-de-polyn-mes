%-----------------------------------------------------------------------%
%______//------ Matrices aléatoires et zéros de polynômes ------\\______%
%______||------                Chapitre 3                ------||______%
%______\\------  ------//______%
%-----------------------------------------------------------------------%

\chapter{Preuves des théorèmes 1 et 2}

\section{Polynômes réels stables}

Notons $\HH = \{z \in \C \colon \im z > 0\}$ le demi-plan supérieur de $\C$.

\begin{defi}
  Un polynômes en $d$ variables $p(z_1, \ldots, z_d)$ est \emph{\index{polynôme réel stable}réel stable} si
  \begin{itemize}
  \item ses coefficients sont réels;
  \item $p$ n'admet aucun zéro dans $\HH^d$.
  \end{itemize}

\end{defi}

\begin{ex}
  Pour $d = 1$, $p(z)$ est réel stables si et seulement si ses coefficients sont réels et tous les zéros de
  $p$ sont réels.
\end{ex}

\begin{prop}
  \label{prop:1}
  Soient $A_1, \ldots, A_d \geq 0$ dans $M_m(\C)$. Alors 
  \[ q(z, z_1, \ldots, z_d) := \det\left( z \un_m + \sum_{i = 1}^{d} z_i A_i \right) \]
  est réel stable.
\end{prop}

\begin{preuve}
  On a 
  \[ q(\overline{z}, \overline{z_1}, \ldots, \overline{z_d}) = \overline{q(z, z_1, \ldots, z_d)} \]
  car $A_i = A_i^\ast$ (i.e. les matrices sont auto-adjointes), donc les coefficients de $q$ sont réels.

  Si $q(z, z_1, \ldots, z_d) = 0$, alors $\det \left( z \un_m + \sum_{i = 1}^{d}z_i A_i\right) = 0$. Ainsi la
  matrice $z \un_m + \sum_{i = 1}^{d} z_i A_i$ n'est pas inversible, et donc le noyau n'est pas vide, i.e. il
  existe $v \in \C^m$ non nul avec $zv + \sum_{i = 1}^{d}z_i A_i v = 0$. On fait le produit scalaire avec $v$, 
  \[ z\underbrace{\|v\|^2}_{> 0} + \sum_{i = 1}^{d}z_i \underbrace{\left \langle A_i v\, ,\, v \right\rangle}_{\geq 0} = 0 \]
  Donc on ne peut pas avoir $\im z > 0$ et $\im z_i > 0$ pour tout $i$, donc $q$ ne s'annule pas dans $\HH^{d
    + 1}$.
\end{preuve}

Une \emph{\index{spécialisation de $p$}spécialisation de $p$} est un polynôme en $d$ variables obtenu en
donnant une valeur fixe à une des variables.

\emph{Attention:} si $p(z_1, z_2) = z_1(z_2-1)$, alors $p(z, 1) \equiv 0$.

\begin{prop}
  \label{prop:2}
  Si $p(z_1, \ldots, z_d)$ est réel stable, en spécialisant $z_d$ en une valeur réelle $t$, on obtient un
  polynôme réel stable, ou le polynôme $0$.
\end{prop}

\begin{preuve}
  Soit $q(z_1, \ldots, z_{d-1}) := p(z_1, \ldots, z_{d-1}, t)$. Les coefficients de $q$ sont réels. On écrit
  \[q(z_1, \ldots, z_{d-1}) = \lim_{n \to \infty} p(z_1, \ldots, z_{d-1}, t + i/n) \]
  uniformément sur tout compact de $\C^{d-1}$. Le théorème de \textsc{Hurwitz} de l'analyse complexe
  s'applique. En effet, $p(z_1, \ldots, z_{d-1}, t + i/n)$ est sans zéro dans $\HH^{d-1}$. Donc la limite est
  sans zéro dans $\HH^{d-1}$ ou identiquement nul.
\end{preuve}


On note $\partial_i = \frac{\partial}{\partial z_i}$ la dérivée partielle par rapport à $z_i$.


\begin{prop}
  \label{prop:3}
  Si $p(z_1, \ldots, z_d)$ est réel stable, alors pour tout $t \in \R$, 
  \[ (1 + t \partial_d)p = p + t \frac{\partial p}{\partial z_d} \]
  est réel stable.
\end{prop}


\begin{preuve}
  Si $t = 0$, c'est bon. Supposons $t \neq 0$. On va procéder par l'absurde. 
  \[ \left( (1 + t\partial_d)p \right)(z_1, \ldots, z_d) = 0 \text{ et } (z_1, \ldots, z_d) \in \HH^{d}.\]
  Soit 
  \[ q(z):= p(z_1, \ldots, z_{d-1}, z). \]
  Alors $q$ n'a pas de zéro dans $\HH$ (sinon $p$ ne serait pas réel stable). En particulier, 
  \[ q(z_d) \neq 0. \]
  Si $n = \mathrm{deg} q$, alors 
  \[ q(z) = \prod_{i = 1}^{n} (z-\omega_i) \tag{$\ast$}\]
  où les $\omega_i$ sont les zéros complexes de $q$. Par ce qui précède, $\im \omega_i < 0$ pour tout
  $i$. Alors, 
  \[ 0 = \left((1 + t \partial_d)p\right)(z_1, \ldots, z_d) = (q + tq')(z_d) = q(z_d) \left(1 + t
      \frac{q'(z_d)}{q(z_d)}\right) \]
  où le terme $\frac{q'(z_d)}{q(z_d)}$ est la \index{dérivée logarithmique}dérivée logarithmique de $q$ en
  $z_d$. Ainsi on obtient 
  \[ 0 = 1 + t \frac{q'(z_d)}{q(z_d)}.\]
  On prend à présent la dérivée logarithmique de $(\ast)$. Alors 
  \[ \frac{q'(z_d)}{q(z_d)} = \sum_{i = 1}^{n} \frac{1}{z- \omega_i}. \]
  On déduit de ceci que (en évaluant en $z_d$)
  \[ 0 = 1 + t \sum_{i = 1}^{n} \frac{1}{z_d - \omega_i} = 1 + t \sum_{i = 1}^{n} \frac{\overline{z_d} -
      \overline{\omega_i}}{|z_d - \omega_i|^2}. \]
  On prend à présent les parties imaginaires: 
  \[ 0 = t \sum_{i = 1}^{n} \frac{\im(\omega_i) - \im(z_d)}{|z_d - \omega_i|^2} \]
  et $\im(\omega_i) < 0$ et $\im(z_d) > 0$, qui est une contradiction car $t \neq 0$ par hypothèse et la somme
  est strictement négative (donc non nulle).
\end{preuve}


Pour $x = (x_1, \ldots, x_d) \in \R^d$, l'\emph{\index{orthant}orthant} $\{y \geq x\}$ est par définition 
\[ \{y \geq d\} := \left\{ (y_1, \ldots, y_d) \in \R^d \colon y_i \geq x_i \ \forall i = 1, \ldots,
    d\right\} \]
(on y pensera comme le quart de plan (en deux dimensions) où les coordonnées de $y$ sont toutes plus grandes
que celles de $x$, ou comme le huitième de plan en trois dimensions).

Pour une fonction $f \colon \R^d \to \C$ de classe $\mathcal{C}^1$, on note $\Phi_f^j = \frac{\partial_j
  f}{f}$ la \index{dérivée logarithmique}dérivée logarithmique de $f$ par rapport à la $j$-ème variable.

\begin{lem}[Lemme 3.7 des notes]
  \label{lem:4}
  Soit $x = (x_1, \ldots, x_d) \in \R^d$, $p(z_1, \ldots, z_d)$ réel stable, sans zéro dans l'orthant
  $\{y \geq x\}$. Supposons qu'il existe $j \in \{1, \ldots, d\}$ tel qu'il existe $\delta > 0$ (penser à
  $\delta$ grand) avec $\Phi_p^j(x_1, \ldots, x_d) \leq 1 - \frac{1}{\delta}$. Alors $(1 - \partial_j)p$ n'a
  pas de zéro dans l'orthant $\{y \geq x + \delta e_j\}$, et de plus, pour tout $i = 1, \ldots, d$,
  \begin{equation}
    \label{eq:lemme-degue}
    \Phi_{(1-\partial_j)p}^i(x + \delta e_j)  \leq \Phi_p^i(x).
  \end{equation}
\end{lem}

\begin{prop}
  \label{prop:5}
  Soit $x = (x_1, \ldots, x_d) \in \R^d$, $p(z_1, \ldots, z_d)$ un polynôme réel stable sans zéro dans
  l'orthant $\{y \geq x\}$. S'il existe $\delta > 0$ tel que 
  \[ \Phi_p^j (x_1, \ldots x_d) \leq 1 - \frac{1}{\delta}\ \forall j = 1, \ldots, d, \]
  alors le polynôme 
  \[ \prod_{i=1}^d (1 - \partial_i)p \]
  est sans zéro dans $\{y \geq x + D\}$ où $D = (\delta, \delta, \ldots, \delta)$.
\end{prop}

\begin{preuve}
  Pour $1 \leq k \leq d$, soit $x^{(k)} = (x_1 + \delta, x_2 + \delta, \ldots, x_k + \delta, x_{k+1}, \ldots,
  x_d)$ et 
  \[ q_k = \prod_{i=1}^k \left(1 - \partial_i\right)p \]
  qui est réel stable par la Proposition \ref{prop:3}. Par récurrence à partir du Lemme \ref{lem:4}, $q_k$ n'a
  pas de zéro dans l'orthant $\{y \geq x^{(k)}\}$ et 
  \[ \Phi_{q_k}^j (x^{(k)}) \leq 1 - \frac{1}{\delta} \]
  pour $j = 1, \ldots, d$. Pour $k = d$, on obtient la proposition.
\end{preuve}



\section{Polynômes caractéristiques mixtes}
\label{sec:polyn-caract-mixtes}

\paragraph{Rappel:} pour $A \in M_m(\C)$, le polynôme caractéristique est 
\[ p_A(z) = \det(z \un_m - A). \]

\begin{defi}
  Pour $A_1, \ldots, A_d \in M_m(\C)$, le \emph{\index{polynôme caractéristique mixte}polynôme caractéristique
    mixte} est 
  \[ \mu[A_1, \ldots, A_d](z) = \left . \left(\prod_{j=1}^{d}(1 - \partial_j) \det\left(z \un_m + \sum_{i=1}^{d} z_i
        A_i\right)\right) \right|_{z_1=z_2=\cdots=z_d=0}. \]
\end{defi}

\begin{rem}
  Si $A_1, \ldots, A_d \geq 0$, par les Propositions \ref{prop:1}, \ref{prop:2} et \ref{prop:3}, $\mu[A_1,
  \ldots, A_d]$ est réel stable.
\end{rem}

\begin{prop}
  \label{prop:6}
  Si $\mathrm{rang}(A_i) = 1$ pour $i = 1, \ldots, d$ avec $A = \sum_{i = 1}^{f}A_i$, alors 
  \[ p_A(z) = \mu[A_1, \ldots, A_d](z). \]
\end{prop}

\begin{preuve}
  La preuve se fait en deux pas.
  \begin{enumerate}
  \item Pour tout $B \in M_m(\C)$, le polynôme $(z_1, \ldots, z_d) \mapsto \det \left( B + \sum_{i = 1}^{d}
      z_i A_i \right)$ est affine-multilinéaire, c'est-à-dire un exposant $\geq 2$ n'apparaît dans aucun
    terme. Ainsi chaque terme est de la forme 
    \[ C z_1^{\epsilon_1} z_2^{\epsilon_2} \cdots z_d^{\epsilon_d} \]
    avec $\epsilon_i \in \{0,1\}$. Ou encore, en chaque variable, on a un polynôme de degré $\leq 1$.

    Voyons-le pour $d = 1$ (puis récurrence facile). Le fait que $\mathrm{rang}(A_1) = 1$ implique que $\dim
    \im A_1 = 1$. On prend une base de $\C^m$ dont le premier vecteur est dans $\im A_1$. Dans cette base, 
    \[ A_1 =
      \begin{pmatrix}
        \ast & \ast & \cdots & \ast\\
        0 & 0 & \cdots & 0\\
        \vdots & \vdots & & \vdots\\
        0 & 0 & \cdots & 0
      \end{pmatrix}
    \]
    En développant $\det(B + z_1 + A_1)$ par rapport à la première ligne, on obtient un polynôme de degré
    inférieur ou égal à 1 en $z_1$ (en effet, $B$ n'introduit que des coefficients constants dans la matrice,
    les $z_i$ étant seulement dans la première ligne, on a la conclusion).

    
  \item Formule de Taylor à $d$ variables. Un polynôme affine-multilinéaire est égal à son développement de
    Taylor d'ordre $(1, 1, \ldots, 1)$. On a
    \[ \det \left( B + \sum_{i=1}^{d}t_i A_i \right) =  \left. \left(\prod_{i=1}^d \left(1 + t_i \partial_i\right) \det
          \left( B + \sum_{i = 1}^{d} z_i A_i \right)\right) \right |_{z_1 = \cdots = z_d = 0}\]
    pour tout $t_1, \ldots t_d \in \R$. On fait $t_1 = \cdots = t_d = -1$, $B = z \un_m$ de sorte que 
    \[ \det \left( z \un_m - \sum_{i = 1}^{d} A_i \right) = p_A(z) = \mu[A_1, \ldots, A_d](z). \]
  \end{enumerate}
\end{preuve}


\begin{ex}
  Soit $p(z_1, z_2) = a_{00} + a_{10} z_1 + a_{01} z_2 + a_{11}z_1 z_2$. Alors 
  \[ \left( (1 + t_1 \partial_1)p \right)(z_1,z_2) = p(z_1, z_2) + t_1(a_{10} + a_{11}z_2), \]
  
  \[ \left( (1 + t_2 \partial_2)(1 + t_1 \partial_1)p \right)(z_1, z_2) = p(z_1, z_2) + t_1(a_{10} +
    a_{11}z_2) + t_2(a_{01} + a_{11}z_1) + t_2t_1a_{11}. \]
  En posant $z_1 = z_2 = 0$, alors on obtient donc dans le développement de Taylor 
  \[ a_{00} + a_{10} z_1 + a_{01} z_2 + a_{11}z_1 z_2. \]
\end{ex}








%%% Local Variables:
%%% mode: latex
%%% TeX-master: "../MAeZP_cours.tex" 
%%% End: