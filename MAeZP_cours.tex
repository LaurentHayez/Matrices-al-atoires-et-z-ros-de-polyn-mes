\documentclass[a4paper, 12pt, usenames, dvipsnames, chapterprefix=true]{scrreprt}

\usepackage[utf8]{inputenc}
\usepackage[T1]{fontenc}
\usepackage{graphicx, wrapfig}
\usepackage{concrete}
\usepackage{eulervm}
\usepackage{fancyhdr}
\usepackage{color, colortbl}
\usepackage{xcolor}
\usepackage{amsmath, amssymb, mathrsfs, amsthm, thmtools, MnSymbol}
\usepackage[framemethod=tikz]{mdframed}
\usepackage{pgf, pgfplots, tikz, pst-solides3d}
\usetikzlibrary{cd} %To draw commutative diagrams
\usetikzlibrary{calc}
\usetikzlibrary{arrows}
\usetikzlibrary{shapes}
\usetikzlibrary{lindenmayersystems, arrows.meta}
\usetikzlibrary{automata}
\usepackage[chapter]{algorithm}
\usepackage{algorithmicx, algpseudocode}
\usepackage{listings}
\usepackage{multicol, multirow}
% the following patch corrects a bug for the closing parenthesis  
\usepackage{etoolbox}
\makeatletter
\patchcmd{\lsthk@SelectCharTable}{%
  \lst@ifbreaklines\lst@Def{`)}{\lst@breakProcessOther)}\fi}{}{}{}
\makeatother
\usepackage{hyperref}
\usepackage{todonotes}
\usepackage{makeidx}
\usepackage[backend=biber, style=alphabetic, sorting=none, bibencoding=utf8]{biblatex}
\usepackage[inline]{enumitem}
\usepackage[french]{babel}
\usepackage{caption, tabu}


\newcommand{\N}{\mathbb{N}}
\newcommand{\Q}{\mathbb{Q}} 
\newcommand{\R}{\mathbb{R}}
\newcommand{\Z}{\mathbb{Z}}
\newcommand{\F}{\mathbb{F}}
\newcommand{\Fa}{\F(A)} 
\newcommand{\C}{\mathbb{C}}
\newcommand{\K}{\mathbb{K}}
\renewcommand{\epsilon}{\varepsilon}
\renewcommand{\phi}{\varphi}
\renewcommand{\emph}{\textbf}
\newcommand{\im}{\mathrm{Im}}
\newcommand{\rev}[2]{\substack{#1\\\downarrow\\ #2}p}
\newcommand{\bsalo}{bounded self-adjoint linear operator }

\DeclareMathOperator{\rk}{rk}
\DeclareMathOperator{\nul}{nul}

\synctex=1

% ---------------- For Cayley graph of free group ----------------- %
\newcount\quadrant
\pgfdeclarelindenmayersystem{cayley}{
  \rule{A -> B [ R [A] [+A] [-A] ]}
  \symbol{R}{ \pgflsystemstep=0.5\pgflsystemstep } 
  \symbol{-}{
    \pgfmathsetcount\quadrant{Mod(\quadrant+1,4)}
    \tikzset{rotate=90}
  }
  \symbol{+}{
    \pgfmathsetcount\quadrant{Mod(\quadrant-1,4)}
    \tikzset{rotate=-90}
  }
  \symbol{B}{
    \draw [dot-cayley] (0,0) -- (\pgflsystemstep,0) 
       node [font=\scriptsize, midway, 
         anchor={270-mod(\the\quadrant,2)*90}, inner sep=.5ex] 
           {\ifcase\quadrant$a$\or$b$\or$a^{-1}$\or$b^{-1}$\fi};
    \tikzset{xshift=\pgflsystemstep}
  }
}
\tikzset{
  dot/.tip={Circle[sep=-1.5pt,length=3pt]}, cayley/.tip={Stealth[]dot[]}
}
% ------ taken from http://tex.stackexchange.com/questions/222881/cayley-graph-of-free-group-in-tikz ----- %



%%%%%%%%	Définitions des environnements de théorèmes	%%%%%%%%
%----- ENVIRONNEMENT POUR LES DÉFINITIONS ----%
\declaretheoremstyle[
  spaceabove=0pt, spacebelow=0pt, headfont=\normalfont\bfseries\scshape,
    notefont=\mdseries, notebraces={(}{)}, headpunct={. }, headindent={},
    postheadspace={ }, postheadspace=4pt, bodyfont=\normalfont, %qed=$$,
    mdframed={
      leftmargin=-5,
      rightmargin=-5,
      middlelinewidth=1pt,
      roundcorner=5pt,
      middlelinecolor=OliveGreen,
      innerlinecolor=OliveGreen,
      outerlinecolor=OliveGreen,
      % apptotikzsetting={\tikzset{mdfbackground/.append style ={
      %       shade, left color=OliveGreen!20, right color = OliveGreen!20}}}
   }
]{defstyle}

\declaretheorem[style=defstyle, numberwithin=chapter, title=Définition]{defi}
%________________________________________________________


%----- ENVIRONNEMENT POUR LES THEOREMES ----%
\declaretheoremstyle[
  spaceabove=0pt, spacebelow=0pt, headfont=\normalfont\bfseries\scshape,
    notefont=\mdseries, notebraces={(}{)}, headpunct={. }, headindent={},
    postheadspace={ }, postheadspace=4pt, bodyfont=\normalfont\itshape, %qed=$$,
    mdframed={
      leftmargin=-5,
      rightmargin=-5,
      middlelinewidth=1pt,
      roundcorner=5pt,
      middlelinecolor=OliveGreen,
      innerlinecolor=OliveGreen,
      outerlinecolor=OliveGreen,
      % apptotikzsetting={\tikzset{mdfbackground/.append style ={
      %       shade, left color=OliveGreen!20, right color = OliveGreen!20}}}
   }
]{thmstyle}
\declaretheorem[style=thmstyle, sibling=defi, title=Théorème]{theo}
\declaretheorem[style=thmstyle, sibling=defi, title=Corollaire]{cor}
\declaretheorem[style=thmstyle, sibling=defi, title=Proposition]{prop}
\declaretheorem[style=thmstyle, sibling=defi, title=Propriétés]{propri}
\declaretheorem[style=thmstyle, sibling=defi, title=Observation]{obs}
\declaretheorem[style=thmstyle, sibling=defi, title=Observations]{obss}
\declaretheorem[style=thmstyle, sibling=defi, title=Lemme]{lem}
\declaretheorem[style=thmstyle, sibling=defi, title=Conséquence]{conseq}
\declaretheorem[style=thmstyle, sibling=defi, title=Conséquences]{conseqs}
%_________________________________________________________

%----- ENVIRONNEMENT POUR LES PREUVES ----%
\declaretheoremstyle[
  spaceabove=0pt, spacebelow=0pt, headfont=\normalfont\bfseries\scshape,
    notefont=\mdseries, notebraces={(}{)}, headpunct={. }, headindent={},
    postheadspace={ }, postheadspace=4pt, bodyfont=\normalfont, 
    mdframed={
      leftmargin=15,
      rightmargin=15,
      hidealllines=true,
      font=\small
   }
]{preuvestyle}

\declaretheorem[style=preuvestyle, numbered = no, title=Preuve, qed=\textcolor{OliveGreen!80}{\qedsymbol}]{preuve}
\declaretheorem[style=preuvestyle, numberwithin=chapter, title=Solution de l'exercice, qed=\textcolor{OliveGreen!80}{\qedsymbol}]{solution}
\declaretheorem[style=preuvestyle, title=Exercice, numberwithin=chapter, qed=\textcolor{OliveGreen!80}{$\spadesuit$}]{exercice}
\declaretheorem[style=preuvestyle, sibling=defi, title=Remarque, qed = \textcolor{OliveGreen!80}{$\clubsuit$}]{rem}
\declaretheorem[style=preuvestyle, sibling=defi, title=Remarques, qed = \textcolor{OliveGreen!80}{$\clubsuit$}]{rems}
%________________________________________________________
%----- ENVIRONNEMENT POUR LES EXEMPLES ----%
\declaretheoremstyle[
  spaceabove=0pt, spacebelow=0pt, headfont=\normalfont\bfseries\scshape,
    notefont=\mdseries, notebraces={(}{)}, headpunct={. }, headindent={},
    postheadspace={ }, postheadspace=4pt, bodyfont=\normalfont, qed=\textcolor{OliveGreen!80}{$\bigstar$},
    mdframed={
      leftmargin=15,
      rightmargin=15,
      font=\small,
      outerlinewidth=1pt,
      innerlinewidth=1pt,
      middlelinewidth=1pt,
      hidealllines=true, leftline=true,
      innerlinecolor=OliveGreen!80,
      outerlinecolor=OliveGreen!80,
      middlelinecolor=White,
   }
]{exstyle}

\declaretheorem[style=exstyle, numberlike=defi, title=Exemple]{ex}
\declaretheorem[style=exstyle, numberlike=defi, title=Exemples]{exs}
%________________________________________________________

% Add only one number to align* environment
\newcommand\numberthis{\addtocounter{equation}{1}\tag{\theequation}}

\addtokomafont{disposition}{\normalfont\bfseries}

\title{\normalfont{\bfseries{Matrices aléatoires et zéros de polynômes:\\ Notes de cours}}}
\author{Enseignant: Alain \textsc{Valette}, Scribe: Laurent \textsc{Hayez}}
\date{Année 2018-2019, semestre de printemps\\ Dernière modification: \today}


\makeindex

\begin{document}


\renewcommand{\labelitemi}{\textbullet}

\tikzset{math3d/.style=
{x= {(-0.353cm,-0.353cm)}, y={(1cm,0cm)}, z={(0cm,1cm)}}}


\maketitle


%Table of contents
\tableofcontents

% Chapter 0:
%-----------------------------------------------------------------------%
%______//------ Matrices aléatoires et zéros de polynômes ------\\______%
%______||------                Chapitre 0                 ------||______%
%______\\------  ------//______%
%-----------------------------------------------------------------------%

\setcounter{chapter}{-1}

\chapter{Résumé}

\begin{center}
  \begin{tikzpicture}[scale=0.8]
    \node[cloud, draw, cloud puffs=10, cloud puff arc=120, aspect=1, inner ysep=1em, text width=2cm, text
    centered, ForestGreen] (1) at (0,0){Dirac\\ Mécanique quantique};
    \node[draw, text width = 5cm, text centered] (2) at (12,0) {Problème de \textsc{Kadison-Singer}, 1959\\
      Analyse fonctionnelle};
    \node[draw, text width = 5cm, text centered] (3) at (12,-5) {Traduction en algèbre linéaire (1970-2002)};
    \node[draw, text width = 5cm, text centered] (4) at (6,-10) {Résolution en 2013 par 3 informaticiens
      (algèbre linéaire + calcul à plusieurs variables + probabilités élémentaires)};

    \node[draw, text width = 5cm, text centered] (5) at (0,-15) {Problèmes de théorie des graphes};
    \node[draw, text width = 5cm, text centered] (6) at (12,-15) {Problèmes d'informatique théorique \og
      problème du voyageur de commerce\fg{}};

    \draw[->,>=latex, line width=0.7pt] (1.east) -- (2.west);
    \draw[->,>=latex, line width=0.7pt] (2.south) -- (3.north);
    \draw[->,>=latex, line width=0.7pt] (3.west) -| (4.north);
    \draw[->,>=latex, line width=0.7pt] (4.east) -| (6.north);
    \draw[->,>=latex, line width=0.7pt] (4.west) -| (5.north) node[midway, above right]{Méthode};
  \end{tikzpicture}
\end{center}





%%% Local Variables:
%%% mode: latex
%%% TeX-master: "../MAeZP_cours.tex" 
%%% End:

% Chapter 1:
%-----------------------------------------------------------------------%
%______//------ Matrices aléatoires et zéros de polynômes ------\\______%
%______||------                Chapitre 1                 ------||______%
%______\\------  ------//______%
%-----------------------------------------------------------------------%

\chapter{De la mécanique quantique à l'analyse fonctionnelle}

\section{Mécanique quantique}

 Paul Adrien Maurice \textsc{Dirac} (1902 - 1984) physicien anglais d'origine valaisanne, a écrit en 1930 les
 \og Principles of quantum mechanics\fg{} (plusieurs fois ré-édités).

 Principes:
 \begin{itemize}
 \item Les états d'un système physique sont représentés par les vecteurs-unités d'un espace de Hilbert $\H$.
   \begin{exs}
     \begin{itemize}
     \item Une particule libre à une dimension: $\H = L^2(\R)$.
       
     \item Une particule libre à deux dimensions: $\H = L^2(\R^2)$.
     \end{itemize}
   \end{exs}
   
 \item Les grandeurs physiques (\og observables\fg{}) sont des opérateurs auto-adjoints $(T = T^\ast)$ sur
   $\H$.
   \begin{exs}
     \begin{itemize}
     \item Sur $L^2(\R)$, l'\emph{\index{opérateur de position}opérateur de position} $P$ est la
       multiplication par $x$ sur $L^2(\R)$.
     \item L'\emph{\index{opérateur de moment}opérateur de moment} (ou impulsion) $Q = \frac{1}{i} \frac{d}{dx}$.
     \end{itemize}
   \end{exs}
   
 \item Ce qu'on peut observer (en laboratoire) est la \emph{probabilité} que la valeur d'une observable sur un
   état donné, soit comprise entre deux valeurs $a$ et $b$.
   \begin{exs}
     \begin{itemize}
     \item Soit $\psi \in L^2(\R)$, $\|\psi\| = 1$, c'est-à-dire 
       \[ \int_{-\infty}^\infty |\psi(x)|^2dx = 1.\]
       La probabilité que la position d'une particule dans l'état $\psi$, soit entre $a$ et $b$ est 
       \[ \int_a^b |\psi(x)|^2 dx. \]
       En effet, le fait que $\|\psi\| = 1$ nous dit que $\psi$ est une densité de probabilité.
     \end{itemize}
   \end{exs}
 \end{itemize}

 Ces trois principes sont parfois appelés les \og \index{axiomes de la mécanique quantique}axiomes de la
 mécanique quantique\fg{}.

 Dans le chapitre 18 de son livre, \og Probability amplitudes\fg{}, Dirac donne une recette pour obtenir ces
 probabilités.

 \begin{defi}
   Deux observables $S$ et $T$ sont dit \emph{\index{observables compatibles}compatibles} si $ST = TS$.
 \end{defi}

 \begin{exs}
   \begin{enumerate}
   \item Les observables $P$ et $Q$ précédemment définis ne sont pas compatibles. En effet, 
     \[ PQ - QP = \frac{-1}{i} Id, \]
     qui est  la \emph{\index{relation d'indétermination de Heisenberg}relation d'indétermination de
       Heisenberg}.
     
   \item Pour une particule libre à deux dimensions, posons $P_x$ l'opérateur de multiplication par la
     première variable $x$ sur $L^2(\R)$ ($P_x f(x,y) = xf(x,y)$) et $P_y$ l'opérateur de multiplication par la
     deuxième variable $y$. Ces deux opérateurs sont compatibles, 
     \[ P_xP_y = P_yP_x. \]
   \end{enumerate}
 \end{exs}

 La recette de Dirac est la suivante:
 \begin{enumerate}
 \item considérer un ensemble maximal d'observables deux à deux compatibles,
 \item spécifier les probabilités associées sur les observables dans un état quantique donné,
 \item étendre ces probabilités à toutes les observables, même non compatibles.
 \end{enumerate}


 \section{$C^\ast$-algèbres}
\label{sec:c-star-algebres}

Si $\H$ est un espace de Hilbert, on note $\bh$ l'\index{espace des opérateurs linéaires bornés}espace des
opérateurs linéaires bornés (donc continus) de $\H$ vers $\H$, avec la norme opérateur 
\[ \|T\| = \sup_{\|x\| \leq 1} \|Tx\|. \]
L'espace $\bh$ est une \emph{\index{algèbre de Banach}algèbre de Banach} 
\[ \|ST\| \leq \|S\|\|T\|. \]
Si $T \in \bh$, $T^\ast \in \bh$ est défini par 
\[ \left \langle T \xi\, ,\, \eta \right \rangle = \left \langle \xi\, ,\, T^\ast \eta \right \rangle  \quad
  \forall \xi, \eta \in \H.\]


\begin{defi}
  \begin{itemize}
  \item Une \emph{\index{sous-algèbre}sous-algèbre} est un sous espace vectoriel qui est stable pour la
    multiplication.
  \item Une \emph{\index{$\ast$-sous-algèbre}$\ast$-sous-algèbre} $A$ a la propriété $T \in A \Rightarrow
    T^\ast \in A$ et est fermée pour la norme opérateur.
  \end{itemize}

\end{defi}

\begin{defi}
  Une \emph{\index{$C^\ast$-algèbre}$C^\ast$-algèbre} est une $\ast$-sous-algèbre fermée de $\bh$.
\end{defi}

\begin{exs}
  \begin{enumerate}
  \item $\bh$, $\C \mathbbm{1} = \{\lambda \mathbbm{1},\ \lambda \in \C\}$, $\{0\}$, $\mathcal{K}(\H)$ l'ensemble des opérateurs
    compacts.
  \item Soit $X$ un espace topologique compact, et $C(X)$ l'ensemble des fonctions continues de $X$ dans $\C$
    muni de la norme 
    \[ \|f\|_{\infty} = \max_{x \in X} |f(x)| \]
    et de l'involution 
    \[ f^\ast(x) = \overline{f(x)}. \]
    On prend sur $X$ une mesure de probabilité $\mu$ telle que $\mu(U) > 0$ pour tout ouvert non vide de
    $X$. \textit{L'exemple a garder en tête est $[0,1]$ avec la mesure de Lebesgue.} Ainsi $\H = L^2(X, \mu)$
    est un espace de Hilbert. Si on multiplie une fonction de $L^2$ par une fonction continue, elle reste dans
    $L^2$, ainsi 
    \[ \pi \colon C(X) \to \bh,\ f \mapsto \text{multiplication par $f$ sur } \H. \]
    \begin{exercice}
      Avec $\|\pi(f)\| = \|f\|_\infty$, $C(X)$ est une $C^\ast$-algèbre.
    \end{exercice}
  \end{enumerate}
\end{exs}

\begin{theo}[Gelfand, 1940]
  Soit $A$ une $C^\ast$-algèbre commutative à unité $(1 \in A)$. Il existe un espace compact $X$, unique à
  homéomorphisme près, tel que $A \simeq C(X)$.
\end{theo}


\begin{defi}
  Un opérateur $T \in \bh$ est \emph{\index{opérateur positif}positif} $(T \geq 0)$ si les conditions équivalentes
  suivantes sont satisfaites:
  \begin{enumerate}
  \item pour tout $\xi \in \H$, $\left \langle T\xi\, ,\, \xi \right \rangle \geq 0$,
  \item il existe $S \in \bh$ tel que $T = S^\ast S$,
  \item $T = T^\ast$ et $\Sp(T) \subset [0, + \infty[$ (Rappel: le spectre de $T$ $\Sp(T) = \{\lambda
      \in \C\ \colon \ T - \lambda \mathbbm{1} \text{ n'est pas inversible}\}$, $\Sp(T)$ est un compact non
      vide de $\C$ et $\Sp(T) \subseteq B(0, \|T\|)$).
  \end{enumerate}
\end{defi}


\begin{defi}
  Si $A$ est une $C^\ast$-algèbre à unité, un \emph{\index{état}état} sur $A$ est une forme linéaire $\phi: A
  \to \C$ telle que
  \begin{enumerate}
  \item $\phi(1) = 1$,
  \item $\phi(T^\ast T) \geq 0$ pour tout $T \in A$.
  \end{enumerate}
  On note $S(A)$ l'ensemble des états sur $A$.
\end{defi}


\begin{ex}
  Soit $\xi \in \H$ tel que $\|\xi\| = 1$, alors pour $T \in \bh$
  \[ \phi(T) = \left \langle T\xi\, ,\, \xi \right \rangle \]
  est un \emph{\index{état vectoriel}état vectoriel} et 
  \[ \phi(1) = \|\xi\|^2 = 1,\ \phi(T^\ast T) = \left \langle T^\ast T \xi\, ,\, \xi \right \rangle  = \left
      \langle T \xi\, ,\, T \xi \right \rangle  = \|T \xi\|^2 \geq 0 \]
  et ainsi $S(A) \neq \emptyset$.
\end{ex}

\begin{prop}
  $S(A)$ est une partie convexe de la boule-unité du dual $A^\ast$ (ici $A^\ast$ est l'ensemble des formes
  linéaires continues sur $A$).
\end{prop}

\begin{preuve}
  \begin{enumerate}
  \item Soient $\phi_1, \phi_2 \in S(A)$, soit $t \in [0,1]$. On doit montrer que 
    \[ t \phi_1 + (1-t)\phi_2 \in S(A). \]
    On a 
    \[ (t \phi_1 + (1-t)\phi_2)(\mathbbm{1}) = 1,\quad (t \phi_1 + (1-t)\phi_2)(T^\ast T) \geq 0. \]
    
  \item Si $\phi \in S(A)$, on doit montrer que $\|\phi\| \leq 1$, c'est-à-dire 
    \[ |\phi(t)| \leq 1 \text{ si } \|T\| \leq 1. \]
    Si $\|T\| \leq 1$, alors $\un - T^\ast T \geq 0$ car 
    \[ \left \langle (\un - T^\ast T)\xi\, ,\, \xi \right \rangle  = \|\xi\|^2 - \|T \xi\|^2 \geq 0 \]
    puisque $\|T\| \leq 1$. On peut encore écrire 
    \[ \un - T^\ast T = S^\ast S \]
    pour $S \in A$. Ainsi 
    \[ 1 - \phi(T^\ast T) = \phi(\un - T^\ast T) = \phi(S^\ast S) \geq 0 \implies \phi(T^\ast T) \leq 1. \]
    L'application $A \times A \to \C$, $(x, y) \mapsto \phi(y^\ast x)$ vérifie l'inégalité de Cauchy-Schwartz 
    \[ |\phi(y^\ast x)|^2 \leq \phi(y^\ast y) \phi(x^\ast x)\, \forall x, y \in A. \]
    Pour $x = T$, $y = \un$, 
    \[ |\phi(T)|^2 \leq \phi(\un) \phi(T^\ast T) \leq 1. \]
  \end{enumerate}
\end{preuve}

\begin{rem}
  Comme $1 = \phi(\un) \leq \|\phi\|$, on a que $S(A)$ est contenu dans la sphère unité de $A^\ast$.
\end{rem}

\begin{defi}
  Soit $K$ un convexe dans un espace vectoriel réel ou complexe. Un point $K$ est \emph{\index{point
      extrême}extrême} dans $K$ si $x$ n'est pas intérieur à un segment contenu dans $K$, c'est-à-dire si $x =
  tx_1 + (1-t)x_2$ avec $0 < t < 1$, $x_1, x_2 \in K$, alors $x = x_1 = x_2$.

  \begin{center}
    \begin{tikzpicture}
      \node[red] (A) at (0,0) {$\bullet$};
      \node[red] (B) at (3, -2) {$\bullet$};
      \node[red] (C) at (-2, -3) {$\bullet$};
      \draw (A.center) -- (B.center) -- (C.center) -- (A.center);
      \draw (A) node[red]{$\bullet$};
      \draw (B) node[red]{$\bullet$};
      \draw (C) node[red]{$\bullet$};
      \draw[ForestGreen, line width=0.7pt] (0, -2) -- (1, -1.5) node[midway]{$\bullet$};
      \draw[ForestGreen, line width=0.7pt] (-1, -1.5) -- (-0.5, -0.75) node[midway]{$\bullet$};
      \draw (8, -1.5) node{\textcolor{red}{$\bullet$} = points extrêmes};
    \end{tikzpicture}
  \end{center}

  Un point extrême de $S(A)$ est
  appelé un \emph{\index{état pur}état pur}.
\end{defi}

\begin{exs}
  \begin{enumerate}
  \item On montre que tout état vectoriel de $\bh$ est pur. On verra qu'il y a des états purs non vectoriels.
  \item Soit $A = C(X)$, alors $S(A)$ s'identifie avec $\mathrm{Prob}(X)$ l'espace des mesures de probabilités
    sur $X$ (un des théorèmes de représentation de \textsc{Riesz}). \qedhere
  \end{enumerate}
\end{exs}

Les états purs s'identifient aux points extrêmes de $\mathrm{Prob}(X)$, c'est-à-dire aux mesures de Dirac
$\delta_x$, définie par (pour $A \subset X$)
\[ \delta_x(A) =
  \begin{cases}
    1 & \text{si } x \in A,\\
    0 & \text{sinon.}
  \end{cases}
\]








%%% Local Variables:
%%% mode: latex
%%% TeX-master: "../MAeZP_cours.tex" 
%%% End:

% Chapter 2: 
%-----------------------------------------------------------------------%
%______//------ Matrices aléatoires et zéros de polynômes ------\\______%
%______||------                Chapitre 2                 ------||______%
%______\\------  ------//______%
%-----------------------------------------------------------------------%

\chapter{De l'analyse fonctionnelle à l'algèbre linéaire}
\label{sec:chap2}


\section{Conjecture de pavage}
\label{sec:conjecture-de-pavage}

En 1979, le mathématicien américain Joel \textsc{Anderson} propose la \emph{\index{conjecture de
    pavage}conjecture de pavage}

\begin{conj}
  Pour tout $\epsilon > 0$, il existe $r \in \N$ tel que pour tout $T \in \B(\ell^2(N))$ (ou $M_m(\C)$) avec
  $\mathrm{diag}(T) = 0$, il existe $Q_1, \ldots, Q_r$ des projecteurs diagonaux avec 
  \[ \sum_{i=1}^{r} Q_i = \mathbbm{1} \]
  et 
  \[ \|Q_iTQ_i\|  \leq \epsilon \|T\|, \ \forall i \in \{1, \ldots, r\}. \]
\end{conj}

Pour rappel, un \index{projecteur}projecteur $P$ est tel que 
\[ P = P^2 = P^\ast. \]
C'est donc une projection orthogonale sur un certain sous-espace fermé de $\H$. Un \index{projecteur
  diagonal}projecteur diagonal a pour coefficients $a_{ii} \in {0, 1}$ et $a_{ij} = 0$ pour $i \neq j$.

Il y a une bijection entre les projecteurs diagonaux et les parties de $\N$. Ainsi 
\[ \sum_{i = 1}^{r} Q_i = \mathbbm{1} \]
signifie qu'on partitionne $\N$ en $r$ parties. Les $Q_1, \ldots, Q_r$ donnent une décomposition de
$\ell^2(\N)$ en $r$ blocs

//Insérer figure matrice ici

La conjecture dit que la condition $\mathrm{diag}T = 0$ implique que les blocs sont de norme petite.

\begin{ex}
Soit $S: \ell^2(\N) \to \ell^2(\N)$ l'\index{opérateur de décalage unilatéral}opérateur de décalage
  unilatéral (\og unilateral shift\fg{}) défini par $Se_n = e_{n+1}$ ou
  $S(a_1, a_2, \ldots) = (0, a_1, a_2, \ldots)$. Alors
  \[ S = \text{insert matrix here} \] On prend la partition de $\N = (2\N) \cup (2\N + 1)$. Soit $Q_1$ la
  projection sur $\ell^2(2\N)$ et $Q_2$ la projection sur $\ell^2 (2\N + 1)$. Alors 
  \[ S =
    \begin{pmatrix}
      0 = Q_1Sq_1 & \ast \\ \ast & 0 = Q_2SQ_2
    \end{pmatrix}
  \]
  L'opérateur $S$ vérifie la conjecture de pavage (\og tiling conjecture\fg{}) avec $\epsilon = 0$ et $r = 2$.
\end{ex}

\begin{prop}
  Si la conjecture de pavage est vraie, alors 
  $ \ell^\infty(\N) $ a l'extension unique des états purs.
\end{prop}

\begin{preuve}
  Soit $\phi$ un état pur de $\ell^\infty(\N)$, $\psi$ une extension de $\phi$ à $\B(\ell^2(\N))$. On veut
  montrer que 
  \[ \psi = \phi \circ \mathrm{diag}. \]
  C'est-à-dire, pour tout $T \in \B(\ell^2(N))$, 
  \[ \psi(T) = \phi(\mathrm{diag}(T)). \]
  En remplaçant $T$ par $T - \mathrm{diag}(T)$, on obtient un opérateur de diagonale nulle. On doit donc
  montrer que si $\mathrm{diag} T = 0$, alors $\psi(T) = 0$, qui est ce qu'on va démontrer.

  On montre que pour tout $\epsilon > 0$, $|\psi(T)| \leq \epsilon \|T\|$. Par la conjecture de pavage, on
  trouve $r \in \N$ et des projecteurs diagonaux $Q_1, \ldots, Q_r$ tels que 
  \[ \sum_{i = 1}^{r} Q_i = \mathbbm{1} \]
  et 
  \[ \|Q_iTQ_i\| < \epsilon \|T\|. \]
  On utilise à présent le fait que $\phi$ est un état pur: $\phi$ est multiplicatif sur $\ell^\infty(\N)$
  (c'est-à-dire que $\phi(ST) = \phi(S)\phi(T)$ si $S, T \in \ell^\infty(\N)$). La raison est que
  $\ell^\infty(\N) = C(X)$ l'espace des fonctions continues sur $X$ car $\ell^\infty$ est une
  $C^\ast$-algèbre commutative à unité (c'est le théorème de \textsc{Gelfand}). Les états purs de $C(X$ sont
  les évaluations aux points, elles sont multiplicatives (ici $X = B\N$ est le compactifié de
  \textsc{Stone-\v{C}ech} de $\N$, c'est la plus grosse compactification de $\N$).
  Alors 
  \[ \phi(Q_i) = \phi(Q_i^2) = \phi(Q_i)^2 \implies \phi(Q_i) \in \{0,1\}. \]
  De plus, 
  \[ 1 = \phi(\mathbbm{1}) = \phi \left(\sum_{i = 1}^{r} Q_i\right) = \sum_{i = 1}^{r}\phi(Q_i). \]
  Puisque $\phi(Q_i) \in \{0,1\}$ et qu'on écrit $1$ comme somme d'éléments de $\{0,1\}$, il existe un unique
  indice $i_0$ avec $\phi(Q_{i_0}) = 1$ et $\phi(Q_i) = 0$ si $i \neq i_0$.

  Alors 
  \[ \psi(T) = \psi \left( \left(\sum_{i=1}^{r} Q_i\right) T \left(\sum_{j = 1}^{r} Q_j\right)\right) =
    \sum_{i,j=1}^{r} \psi(Q_iTQ_j). \]
  Si on sait que dans ces $r^2$ termes, le seul terme non nul est $\psi(Q_{i_0}TQ_{i_0})$, alors 
  \[ |\psi(T)| = |\psi(Q_{i_0}TQ_{i_0})| \leq \|Q_{i_0}TQ_{i_0}\| \leq \epsilon \|T\| \]
  comme on voulait, où la dernière inégalité suit de la conjecture de pavage, et la première du fait que
  $\psi$ est un état pur, donc de norme $1$ (note: à vérifier).

  Par Cauchy-Schwartz, 
  \begin{align*}
    |\psi(Q_iTQ_j)| &= |\psi\left((T^\ast Q_i)^\ast Q_j\right)\\
    &\leq \psi \left((T^\ast Q_i)^\ast (T^\ast Q_i)\right)^{1/2} \underbrace{\psi(\underbrace{Q_j^\ast Q_j}_{=Q_j}}_{\phi(Q_j)}) ^{1/2}\\
    &= 0 \text{ si } j \neq i_0.
  \end{align*}
  De même, 
  \[ |\psi(Q_i T Q_j)| = 0 \]
  si $i \neq i_0$ par un argument similaire. Le seul terme non nul restant est donc $\psi(Q_{i_0}TQ_{i_0})$,
  comme on le souhaitait.
\end{preuve}



\section{Conjecture de \textsc{Weaver} (2004)}
\label{sec:conj-de-weaver}
(Approche de Terry \textsc{Tao}, 2013)

\begin{conj}[\index{conjecture de \textsc{Weaver}}conjecture de \textsc{Weaver}]
  On fixe des entiers $d, m, r \geq 2$ et une constante $c > 0$. Soient $A_1, \ldots, A_d \in M_m(\C)$, $A_i
  \geq 0$, $\mathrm{rang}(A_i) = 1$ avec $\|A_i\| \leq C$ pour tout $i = 1, \ldots d$ et 
  \[ \sum_{i = 1}^{d}A_i = \mathbbm{1}_m. \]
  Alors il existe une partition $\{S_1, S_2, \ldots, S_r\}$ de $\{1, \ldots, d\}$ telle que 
  \[ \left\| \sum_{i \in S_j}^{} A_i \right\| \leq \left(\sqrt{\frac{1}{r}} + \sqrt{C}\right)^2 \]
  pour $j = 1, \ldots, r$.
\end{conj}

Pour cette conjecture, il faut penser à $d$ et $m$ grands, et puisque les matrices sont de rang $1$ et qu'on
veut avoir leur somme égale à $\mathbbm{1}$, on a besoin d'au moins $m$ telle matrices, i.e., $d \geq m$. De
plus il faut penser à $r$ petit (cas extrême, $r = 2$).

Si $A \geq 0$ et $\mathrm{rang}(A) = 1$ alors $A$ est un multiple positif d'un projecteur orthogonal de rang
$1$. C'est-à-dire qu'il existe $\xi \in \C^m$ tel que $\|\xi\| = 1$ et $\lambda > 0$ tel que 
\[ A(v) = \lambda \left \langle v\, ,\, \xi \right \rangle \xi. \]

Si les $A_i$ ont des images $2$ à $2$ orthogonales (sans supposer que leur somme vaut $\mathbbm{1}$ comme dans
la conjecture), alors la somme se décompose par blocs, 
\[ \sum_{}^{} A_i =
  \begin{pmatrix}
    A_1 & 0 & \cdots & 0\\ 0 & A_2 &  & 0\\ \vdots & \vdots & \ddots & \vdots \\ 0 & 0 & \cdots & A_k
  \end{pmatrix} \implies \left \| \sum_{}^{} A_i \right \| = \max \|A_i\| \leq C.
  \]

Ainsi, ce que veut dire la conjecture, c'est que ces matrices sont \textit{presque} orthogonales. En d'autres
termes, on peut partitionner l'ensemble des indices de façon à ce que pour chaque classe de la partition, les
$A_i$ soient d'images \textit{quasiment} orthogonales.

\begin{prop}
  La conjecture de \textsc{Weaver} implique la conjecture de pavage.
\end{prop}







%%% Local Variables:
%%% mode: latex
%%% TeX-master: "../MAeZP_cours.tex" 
%%% End:

% % Chapter 4: Propriétés du groupe libre
% \input{Chapters/GAC_cours-chap4.tex}

% % Chapter 5: Introduction à la topologie algébrique
% \input{Chapters/GAC_cours-chap5.tex}

% % Chapter 6: Transformations de Tietze
% \input{Chapters/GAC_cours-chap6.tex}

% % Chapter 7: Graphes de Cayley
% \input{Chapters/GAC_cours-chap7.tex}

% % Chapter 8: Propriétés géométriques
% \input{Chapters/GAC_cours-chap8.tex}

% % Chapter 8: Croissance et langages formels
% \input{Chapters/GAC_cours-chap9.tex}


\printindex


\appendix
	
\end{document}



%%% Local Variables:
%%% mode: latex
%%% TeX-master: t 
%%% End: